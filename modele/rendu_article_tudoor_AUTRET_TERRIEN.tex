%% bare_jrnl_compsoc.tex
%% V1.4b
%% 2015/08/26
%% by Michael Shell
%% See:
%% http://www.michaelshell.org/
%% for current contact information.
%%
%% This is a skeleton file demonstrating the use of IEEEtran.cls
%% (requires IEEEtran.cls version 1.8b or later) with an IEEE
%% Computer Society journal paper.
%%
%% Support sites:
%% http://www.michaelshell.org/tex/ieeetran/
%% http://www.ctan.org/pkg/ieeetran
%% and
%% http://www.ieee.org/

%%*************************************************************************
%% Legal Notice:
%% This code is offered as-is without any warranty either expressed or
%% implied; without even the implied warranty of MERCHANTABILITY or
%% FITNESS FOR A PARTICULAR PURPOSE! 
%% User assumes all risk.
%% In no event shall the IEEE or any contributor to this code be liable for
%% any damages or losses, including, but not limited to, incidental,
%% consequential, or any other damages, resulting from the use or misuse
%% of any information contained here.
%%
%% All comments are the opinions of their respective authors and are not
%% necessarily endorsed by the IEEE.
%%
%% This work is distributed under the LaTeX Project Public License (LPPL)
%% ( http://www.latex-project.org/ ) version 1.3, and may be freely used,
%% distributed and modified. A copy of the LPPL, version 1.3, is included
%% in the base LaTeX documentation of all distributions of LaTeX released
%% 2003/12/01 or later.
%% Retain all contribution notices and credits.
%% ** Modified files should be clearly indicated as such, including  **
%% ** renaming them and changing author support contact information. **
%%*************************************************************************


% *** Authors should verify (and, if needed, correct) their LaTeX system  ***
% *** with the testflow diagnostic prior to trusting their LaTeX platform ***
% *** with production work. The IEEE's font choices and paper sizes can   ***
% *** trigger bugs that do not appear when using other class files.       ***                          ***
% The testflow support page is at:
% http://www.michaelshell.org/tex/testflow/


\documentclass[10pt,journal,compsoc]{IEEEtran}
%
% If IEEEtran.cls has not been installed into the LaTeX system files,
% manually specify the path to it like:
% \documentclass[10pt,journal,compsoc]{../sty/IEEEtran}

% Police par défaut pour éviter les erreurs
\usepackage{mathptmx}

% Package pour le texte fictif latin
\usepackage{lipsum}


% Some very useful LaTeX packages include:
% (uncomment the ones you want to load)


% *** MISC UTILITY PACKAGES ***
%
%\usepackage{ifpdf}
% Heiko Oberdiek's ifpdf.sty is very useful if you need conditional
% compilation based on whether the output is pdf or dvi.
% usage:
% \ifpdf
%   % pdf code
% \else
%   % dvi code
% \fi
% The latest version of ifpdf.sty can be obtained from:
% http://www.ctan.org/pkg/ifpdf
% Also, note that IEEEtran.cls V1.7 and later provides a builtin
% \ifCLASSINFOpdf conditional that works the same way.
% When switching from latex to pdflatex and vice-versa, the compiler may
% have to be run twice to clear warning/error messages.






% *** CITATION PACKAGES ***
%
\ifCLASSOPTIONcompsoc
  % IEEE Computer Society needs nocompress option
  % requires cite.sty v4.0 or later (November 2003)
  \usepackage[nocompress]{cite}
\else
  % normal IEEE
  \usepackage{cite}
\fi





% *** GRAPHICS RELATED PACKAGES ***
%
\ifCLASSINFOpdf
  % \usepackage[pdftex]{graphicx}
  % declare the path(s) where your graphic files are
  % \graphicspath{{../pdf/}{../jpeg/}}
  % and their extensions so you won't have to specify these with
  % every instance of \includegraphics
  % \DeclareGraphicsExtensions{.pdf,.jpeg,.png}
\else
  % or other class option (dvipsone, dvipdf, if not using dvips). graphicx
  % will default to the driver specified in the system graphics.cfg if no
  % driver is specified.
  % \usepackage[dvips]{graphicx}
  % declare the path(s) where your graphic files are
  % \graphicspath{{../eps/}}
  % and their extensions so you won't have to specify these with
  % every instance of \includegraphics
  % \DeclareGraphicsExtensions{.eps}
\fi
% graphicx was written by David Carlisle and Sebastian Rahtz. It is
% required if you want graphics, photos, etc. graphicx.sty is already
% installed on most LaTeX systems. The latest version and documentation
% can be obtained at: 
% http://www.ctan.org/pkg/graphicx
% Another good source of documentation is "Using Imported Graphics in
% LaTeX2e" by Keith Reckdahl which can be found at:
% http://www.ctan.org/pkg/epslatex
%
% latex, and pdflatex in dvi mode, support graphics in encapsulated
% postscript (.eps) format. pdflatex in pdf mode supports graphics
% in .pdf, .jpeg, .png and .mps (metapost) formats. Users should ensure
% that all non-photo figures use a vector format (.eps, .pdf, .mps) and
% not a bitmapped formats (.jpeg, .png). The IEEE frowns on bitmapped formats
% which can result in "jaggedy"/blurry rendering of lines and letters as
% well as large increases in file sizes.
%
% You can find documentation about the pdfTeX application at:
% http://www.tug.org/applications/pdftex


\usepackage{graphicx}

\hyphenation{op-tical net-works semi-conduc-tor}


\begin{document}
%
% paper title
% Titles are generally capitalized except for words such as a, an, and, as,
% at, but, by, for, in, nor, of, on, or, the, to and up, which are usually
% not capitalized unless they are the first or last word of the title.
% Linebreaks \\ can be used within to get better formatting as desired.
% Do not put math or special symbols in the title.
\title{Analyzing Logic Vulnerabilities in DNS Response Pre-processing: \\From Kaminsky to TuDoor}

\author{Autret~Lucas
        and~Terrien~Maxime}% <-this % stops a space}

\IEEEtitleabstractindextext{%
\begin{IEEEkeywords}
DNS Security, Logic Vulnerabilities, TuDoor Attack, Kaminsky Attack, SAD DNS, DNS Response Pre-processing
\end{IEEEkeywords}}


% make the title area
\maketitle


\IEEEdisplaynontitleabstractindextext
\IEEEpeerreviewmaketitle


%% ==========================================
%% 1. INTRODUCTION
%% ==========================================
\section{Introduction}\label{sec:introduction}

% Contexte : Le DNS, infrastructure critique mais vieille
The Domaine Name System (DNS) is one of the most critical infrastructure of the modern Internet
because of it's fonction. Designed in the 1980s, this protocol translates human-readable
domain names into IP addresses, making web navgiation easier for users. However, its age and
widespread adoption have made it a prime target for attackers seeking to compromise Internet
communications.

% Problématique : Complexité croissante = nouvelles failles
Over the past two decades, DNS has been the subject of numerous cache poisoning attacks.
The Kaminsky attack in 2008 revealed fundamental weaknesses in the protocol, leading to
multiple patches including source port randomization. Despite these countermeasures, SAD
DNS in 2020 demonstrated that side-channel vulnerabilities in operating systems could bypass
existing protections. More recently, the TuDoor attack (2024) has unveiled a new attack surface:
logic vulnerabilities in DNS response pre-processing, where inconsistent handling of malformed
packets across implementations creates exploitable conditions.

% Annonce du plan
This paper analyzes the evolution of DNS attacks and examines in detail the TuDoor attack
methodology notably on cache poisoning. We will first present the DNS architecture in 
section~\ref{sec:context} to get a better understanding of how it works. Then we will get an 
overview of the history of DNS attacks with Kaminsky and SAD DNS. 
After that, we will describe the TuDoor attack in section~\ref{sec:contribution}, its technical
mechanisms, and comparative analysis with prior work. Finally, we will discuss the impact on
DNS security and propose mitigation strategies in section~\ref{sec:discussion}. 

%% ==========================================
%% 2. ÉTAT DE L'ART & HISTORIQUE (~1 page = 1/4 du rapport)
%% ==========================================
\section{State of the Art and Historical Overview}\label{sec:context}

\subsection{How DNS Works}

The Domain Name System (DNS) serves as a crucial component of the Internet infrastructure, that translates 
human-readable domain names into machine-readable IP addresses. As illustrated in Figure~\ref{fig:dns}, the 
resolution process relies on a chain of interactions between several distinct components to locate the correct resource.

The resolution process begins when a client application, such as a web browser, needs to resolve a hostname.
It uses the operating system's \textit{stub resolver} to initiate a request. To optimize performance and reduce the 
latencty, this request is first sent to a pre-configured \textbf{DNS Forwarder}, often integrated into local network 
devices like home Wi-Fi routers. If the forwarder does not have the answenr in its cache, it forwards the query to a 
recursive resolver for further processing.

\begin{figure}[!ht]
  \centering
  \includegraphics[width=\linewidth]{dns.png}
  \caption{General DNS resolver roles and domain name resolution process.}
  \label{fig:dns}
\end{figure}

The \textbf{Recursive Resolver} plays an impoirtant role in the DNS resolution process. Upon receiving a query from the
forwarder, it first checks its cache for a valid response. If the answer is not cached, the recursive resolver embarks
on a systematic process to resolve the domain name. It begins by querying the \textbf{Root DNS Servers}, 
which provide referrals to the appropriate \textbf{Top-Level Domain (TLD) Servers} based on the domain's 
extension (e.g., .com, .org). The recursive resolver then queries the TLD servers, which in turn refer it to the
\textbf{Authoritative DNS Servers} responsible for the specific domain. Finally, the authoritative server provides the 
requested IP address, which is relayed back through the chain to the original client.

\subsection{Kaminsky attack (2008)}

\lipsum[7]

\lipsum[8]

\lipsum[9]

\lipsum[10]

\subsection{SAD DNS (2020)}

The \textbf{SAD DNS (Side-channel AttackeD DNS) attack}, disclosed in 2020 by researchers from Tsinghua University
and the University of California, Riverside, marked a critical regression in DNS security. It demonstrated a method
to effectively resurrect the classic DNS cache poisoning attack by bypassing the primary mitigation implemented
after the 2008 Kaminsky attack: \textbf{Source Port Randomization (SPR)}.

The success of SAD DNS relies on exploiting a subtle, yet pervasive, vulnerability in the networking stacks of modern
operating systems: the predictable rate limit applied to outgoing \textbf{Internet Control Message Protocol (ICMP)}
error messages, specifically the "Port Unreachable" message. This ICMP rate limit serves as a timing side-channel that
allows an off-path attacker to significantly reduce the entropy of a DNS query.

Prior to this attack, SPR had increased query entropy from 16 bits (Transaction ID, $TxID$) to 32 bits
($TxID$ plus the random 16-bit source port). The attack uses the following sequence to infer the source port:

\begin{enumerate}
    \item \textbf{Probe Emission:} The attacker sends a large burst of spoofed UDP probe packets
    targeting the victim DNS recursive resolver's port range. The source IP address of these probes
    is spoofed to that of the target authoritative name server.
    \item \textbf{ICMP Trigger:} The resolver's kernel generates an ICMP "Port Unreachable" error message whenever
    a probe hits a closed port. Conversely, if the probe hits the active, open port currently used for the pending
    DNS query, the ICMP error is suppressed.
    \item \textbf{Rate Limit Inference:} The key exploitation mechanism is the fact that the operating system applies
    a global rate limit to all outgoing ICMP errors. The attacker sends a final, "unspoofed" probe to a known
    closed port on the resolver, observing the response time.
    \begin{itemize}
        \item If the preceding burst of spoofed probes hit enough closed ports to deplete the global ICMP quota,
        the final legitimate probe will experience response delay or suppression.
        \item If the burst included a hit on the active DNS source port, the corresponding ICMP error was suppressed,
        leaving the global quota available.
    \end{itemize}
    \item \textbf{Source Port Derandomization:} By analyzing the timing and successful delivery of the final probe, 
    the attacker can systematically infer which ports in the range are currently active. This process effectively
    derandomizes the 16-bit source port.
    \item \textbf{Cache Poisoning:} With the source port identified, the remaining entropy is reduced to the 16-bit $TxID$,
    enabling the attacker to easily brute-force the remaining field and inject a definitive, malicious DNS response that
    is accepted by the resolver.
\end{enumerate}

\end{document}

%% ==========================================
%% 3. TUDOOR : L'ATTAQUE LOGIQUE
%% ==========================================
\section{TuDoor Attack}\label{sec:contribution}

\subsection{TuDoor Attack Overview}

\lipsum[15]

\lipsum[16]

\lipsum[17]

\subsection{Technical Details}

\lipsum[18]

\lipsum[19]

\lipsum[20]

\subsection{Mecanism of the Vulnerability}

\lipsum[21]

\lipsum[22]

\lipsum[23]

\subsection{Comparative Analysis with Previous Attacks}

\lipsum[24]

\lipsum[25]

%% ==========================================
%% 4. DISCUSSION & CONCLUSION
%% ==========================================
\section{Discussion and Conclusion}\label{sec:discussion}

\subsection{Impact on DNS Security}

\lipsum[26]

\subsection{Causes and Mitigations}

\lipsum[27]

\subsection{Conclusion}

\lipsum[28]


\begin{thebibliography}{1}
\bibitem{tudoor}
X.~Li \emph{et al.}, ``TuDoor Attack: Systematically Exploring and Exploiting Logic Vulnerabilities in DNS Response Pre-processing with Malformed Packets,'' \emph{2024 IEEE Symposium on Security and Privacy (SP)}, pp. 4459-4477, 2024.

\bibitem{kaminsky}
D.~Kaminsky, ``DNS Vulnerability,'' Black Hat USA, 2008.

\bibitem{saddns}
K.~Qian \emph{et al.}, ``SAD DNS: Exploiting Weakened Trust in DNS,'' \emph{ACM CCS}, 2020.

\bibitem{IEEEhowto:kopka}
H.~Kopka and P.~W. Daly, \emph{A Guide to \LaTeX}, 3rd~ed.\hskip 1em plus
  0.5em minus 0.4em\relax Harlow, England: Addison-Wesley, 1999.
\end{thebibliography}

\end{document}