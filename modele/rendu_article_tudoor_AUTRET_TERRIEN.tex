%% bare_jrnl_compsoc.tex
%% V1.4b
%% 2015/08/26
%% by Michael Shell
%% See:
%% http://www.michaelshell.org/
%% for current contact information.
%%
%% This is a skeleton file demonstrating the use of IEEEtran.cls
%% (requires IEEEtran.cls version 1.8b or later) with an IEEE
%% Computer Society journal paper.
%%
%% Support sites:
%% http://www.michaelshell.org/tex/ieeetran/
%% http://www.ctan.org/pkg/ieeetran
%% and
%% http://www.ieee.org/

%%*************************************************************************
%% Legal Notice:
%% This code is offered as-is without any warranty either expressed or
%% implied; without even the implied warranty of MERCHANTABILITY or
%% FITNESS FOR A PARTICULAR PURPOSE! 
%% User assumes all risk.
%% In no event shall the IEEE or any contributor to this code be liable for
%% any damages or losses, including, but not limited to, incidental,
%% consequential, or any other damages, resulting from the use or misuse
%% of any information contained here.
%%
%% All comments are the opinions of their respective authors and are not
%% necessarily endorsed by the IEEE.
%%
%% This work is distributed under the LaTeX Project Public License (LPPL)
%% ( http://www.latex-project.org/ ) version 1.3, and may be freely used,
%% distributed and modified. A copy of the LPPL, version 1.3, is included
%% in the base LaTeX documentation of all distributions of LaTeX released
%% 2003/12/01 or later.
%% Retain all contribution notices and credits.
%% ** Modified files should be clearly indicated as such, including  **
%% ** renaming them and changing author support contact information. **
%%*************************************************************************


% *** Authors should verify (and, if needed, correct) their LaTeX system  ***
% *** with the testflow diagnostic prior to trusting their LaTeX platform ***
% *** with production work. The IEEE's font choices and paper sizes can   ***
% *** trigger bugs that do not appear when using other class files.       ***                          ***
% The testflow support page is at:
% http://www.michaelshell.org/tex/testflow/


\documentclass[10pt,journal,compsoc]{IEEEtran}
%
% If IEEEtran.cls has not been installed into the LaTeX system files,
% manually specify the path to it like:
% \documentclass[10pt,journal,compsoc]{../sty/IEEEtran}

% Police par défaut pour éviter les erreurs
\usepackage{mathptmx}

% Package pour le texte fictif latin
\usepackage{lipsum}


% Some very useful LaTeX packages include:
% (uncomment the ones you want to load)


% *** MISC UTILITY PACKAGES ***
%
%\usepackage{ifpdf}
% Heiko Oberdiek's ifpdf.sty is very useful if you need conditional
% compilation based on whether the output is pdf or dvi.
% usage:
% \ifpdf
%   % pdf code
% \else
%   % dvi code
% \fi
% The latest version of ifpdf.sty can be obtained from:
% http://www.ctan.org/pkg/ifpdf
% Also, note that IEEEtran.cls V1.7 and later provides a builtin
% \ifCLASSINFOpdf conditional that works the same way.
% When switching from latex to pdflatex and vice-versa, the compiler may
% have to be run twice to clear warning/error messages.






% *** CITATION PACKAGES ***
%
\ifCLASSOPTIONcompsoc
  % IEEE Computer Society needs nocompress option
  % requires cite.sty v4.0 or later (November 2003)
  \usepackage[nocompress]{cite}
\else
  % normal IEEE
  \usepackage{cite}
\fi





% *** GRAPHICS RELATED PACKAGES ***
%
\ifCLASSINFOpdf
  % \usepackage[pdftex]{graphicx}
  % declare the path(s) where your graphic files are
  % \graphicspath{{../pdf/}{../jpeg/}}
  % and their extensions so you won't have to specify these with
  % every instance of \includegraphics
  % \DeclareGraphicsExtensions{.pdf,.jpeg,.png}
\else
  % or other class option (dvipsone, dvipdf, if not using dvips). graphicx
  % will default to the driver specified in the system graphics.cfg if no
  % driver is specified.
  % \usepackage[dvips]{graphicx}
  % declare the path(s) where your graphic files are
  % \graphicspath{{../eps/}}
  % and their extensions so you won't have to specify these with
  % every instance of \includegraphics
  % \DeclareGraphicsExtensions{.eps}
\fi
% graphicx was written by David Carlisle and Sebastian Rahtz. It is
% required if you want graphics, photos, etc. graphicx.sty is already
% installed on most LaTeX systems. The latest version and documentation
% can be obtained at: 
% http://www.ctan.org/pkg/graphicx
% Another good source of documentation is "Using Imported Graphics in
% LaTeX2e" by Keith Reckdahl which can be found at:
% http://www.ctan.org/pkg/epslatex
%
% latex, and pdflatex in dvi mode, support graphics in encapsulated
% postscript (.eps) format. pdflatex in pdf mode supports graphics
% in .pdf, .jpeg, .png and .mps (metapost) formats. Users should ensure
% that all non-photo figures use a vector format (.eps, .pdf, .mps) and
% not a bitmapped formats (.jpeg, .png). The IEEE frowns on bitmapped formats
% which can result in "jaggedy"/blurry rendering of lines and letters as
% well as large increases in file sizes.
%
% You can find documentation about the pdfTeX application at:
% http://www.tug.org/applications/pdftex


\usepackage{graphicx}
\usepackage[center]{caption}

\hyphenation{op-tical net-works semi-conduc-tor}


\begin{document}
%
% paper title
% Titles are generally capitalized except for words such as a, an, and, as,
% at, but, by, for, in, nor, of, on, or, the, to and up, which are usually
% not capitalized unless they are the first or last word of the title.
% Linebreaks \\ can be used within to get better formatting as desired.
% Do not put math or special symbols in the title.
\title{Analyzing Logic Vulnerabilities in DNS Response Pre-processing: \\From Kaminsky to TuDoor}

\author{Autret~Lucas
        and~Terrien~Maxime}% <-this % stops a space}

\IEEEtitleabstractindextext{%
\begin{IEEEkeywords}
DNS Security, Logic Vulnerabilities, TuDoor Attack, Kaminsky Attack, SADDNS, DNS Response Pre-processing
\end{IEEEkeywords}}


% make the title area
\maketitle


\IEEEdisplaynontitleabstractindextext
\IEEEpeerreviewmaketitle


%% ==========================================
%% 1. INTRODUCTION
%% ==========================================
\section{Introduction}\label{sec:introduction}

% Contexte : Le DNS, infrastructure critique mais vieille
The Domaine Name System (DNS) is one of the most critical infrastructure of the modern Internet
because of it's fonction. Designed in the 1980s, this protocol translates human-readable
domain names into IP addresses, making web navgiation easier for users. However, its age and
widespread adoption have made it a prime target for attackers seeking to compromise Internet
communications.

% Problématique : Complexité croissante = nouvelles failles
Over the past two decades, DNS has been the subject of numerous cache poisoning attacks.
The Kaminsky attack in 2008 revealed fundamental weaknesses in the protocol, leading to
multiple patches including source port randomization. Despite these countermeasures, SADDNS
in 2020 demonstrated that side-channel vulnerabilities in operating systems could bypass
existing protections. More recently, the TuDoor attack (2024) has unveiled a new attack surface:
logic vulnerabilities in DNS response pre-processing, where inconsistent handling of malformed
packets across implementations creates exploitable conditions.

% Annonce du plan
This paper analyzes the evolution of DNS attacks and examines in detail the TuDoor attack
methodology notably on cache poisoning. We will first present the DNS architecture in 
section~\ref{sec:context} to get a better understanding of how it works. Then we will get an 
overview of the history of DNS attacks with Kaminsky and SADDNS. 
After that, we will describe the TuDoor attack in section~\ref{sec:contribution}, its technical
mechanisms, and comparative analysis with prior work. Finally, we will discuss the impact on
DNS security and propose mitigation strategies in section~\ref{sec:discussion}. 

%% ==========================================
%% 2. ÉTAT DE L'ART & HISTORIQUE (~1 page = 1/4 du rapport)
%% ==========================================
\section{State of the Art and Historical Overview}\label{sec:context}

\subsection{How DNS Works}

The Domain Name System (DNS) serves as a crucial component of the Internet infrastructure, that translates 
human-readable domain names into machine-readable IP addresses. As illustrated in Figure 1, the 
resolution process relies on a chain of interactions between several distinct components to locate the correct resource.

The resolution process begins when a client application, such as a web browser, needs to resolve a hostname.
It uses the operating system's \textit{stub resolver} to initiate a request. To optimize performance and reduce the 
latencty, this request is first sent to a pre-configured \textbf{DNS Forwarder}, often integrated into local network 
devices like home Wi-Fi routers. If the forwarder does not have the answenr in its cache, it forwards the query to a 
recursive resolver for further processing.

\begin{figure}[!ht]
  \centering
  \includegraphics[width=\linewidth]{dns.png}
  \caption{General DNS resolver roles and domain name resolution process.}
  \label{fig:dns}
\end{figure}

The \textbf{Recursive Resolver} plays an important role in the DNS resolution process. Upon receiving a query from the
forwarder, it first checks its cache for a valid response. If the answer is not cached, the recursive resolver embarks
on a systematic process to resolve the domain name. It begins by querying the \textbf{Root DNS Servers}, 
which provide referrals to the appropriate \textbf{Top-Level Domain (TLD) Servers} based on the domain's 
extension (e.g., .com, .org). The recursive resolver then queries the TLD servers, which in turn refer it to the
\textbf{Authoritative DNS Servers} responsible for the specific domain. Finally, the authoritative server provides the 
requested IP address, which is relayed back through the chain to the original client.

\subsection{Kaminsky attack (2008)}

\begin{figure}[!ht]
  \centering
  \includegraphics[width=\linewidth]{kaminsky.png}
  \caption{Kaminsky attack overview.}
  \label{fig:kaminsky}
\end{figure}


In 2008, security researcher Dan Kaminsky unveiled a critical vulnerability in the DNS protocol. The core issue 
was not just the lack of entropy in the $TxID$, but a technique that allowed an attacker to bypass 
the Time-To-Live (TTL) mechanism that was supposed to slow down cache poisoning attempts.

Prior to this discovery, if an attacker failed to poison a DNS cache, they would have to wait for the TTL of the 
cached record to expire before trying again. Kaminsky's attack exploited the fact that DNS resolvers would 
accept random queries for non-existent subdomains of a target domain (e.g., random123.target.com, random456.target.com).
Since these subdomains do not exist in the cache, the recursive resolver is forced to query the authoritative nameserver,
giving the attacker an infinite number of opportunities to flood the resolver with spoofed responses.

At the time, resolvers typically used a static source port for outgoing DNS queries, which meant that the only
field an attacker needed to guess was the 16-bit $TxID$. This provided a search space of only $2^{16}$ (65,536) possible values.
By sending a large number of spoofed DNS responses with different $TxID$ values, the attacker could eventually
guess the correct one and successfully poison the cache in a matter of minutes.

Ultimately, the malicious response would be accepted by the resolver, which would then cache the incorrect mapping.
This allowed the attacker to redirect users to malicious sites, intercept sensitive information, or launch further attacks.
This discovery led to the implementation of \textbf{Source Port Randomization (SPR)} which increased the entropy
to roughly 32 bits ($TxID$ + random 16-bit source port), making brute-forcing attacks significantly more difficult.

\subsection{SADDNS (2020)}

The \textbf{SADDNS (Side-channel AttackeD DNS) attack}, disclosed in 2020 by researchers from Tsinghua University
and the University of California, Riverside, marked a critical regression in DNS security. It demonstrated a method
to effectively resurrect the classic DNS cache poisoning attack by bypassing the primary mitigation implemented
after the 2008 Kaminsky attack: \textbf{Source Port Randomization (SPR)}.

The success of SADDNS relies on exploiting a subtle, yet pervasive, vulnerability in the networking stacks of modern
operating systems: the predictable rate limit applied to outgoing \textbf{Internet Control Message Protocol (ICMP)}
error messages, specifically the "Port Unreachable" message. This ICMP rate limit serves as a timing side-channel that
allows an off-path attacker to significantly reduce the entropy of a DNS query.

The attack uses the following sequence to infer the source port:

\begin{enumerate}
    \item \textbf{Probe Emission:} The attacker sends a large burst of spoofed UDP probe packets
    targeting the victim DNS recursive resolver's port range. The source IP address of these probes
    is spoofed to that of the target authoritative name server.
    \item \textbf{ICMP Trigger:} The resolver's kernel generates an ICMP "Port Unreachable" error message whenever
    a probe hits a closed port. Conversely, if the probe hits the active, open port currently used for the pending
    DNS query, the ICMP error is suppressed.
    \item \textbf{Rate Limit Inference:} The key exploitation mechanism is the fact that the operating system applies
    a global rate limit to all outgoing ICMP errors. The attacker sends a final, "unspoofed" probe to a known
    closed port on the resolver, observing the response time.
    \begin{itemize}
        \item If the preceding burst of spoofed probes hit enough closed ports to deplete the global ICMP quota,
        the final legitimate probe will experience response delay or suppression.
        \item If the burst included a hit on the active DNS source port, the corresponding ICMP error was suppressed,
        leaving the global quota available.
    \end{itemize}
    \item \textbf{Source Port Derandomization:} By analyzing the timing and successful delivery of the final probe, 
    the attacker can systematically infer which ports in the range are currently active. This process effectively
    derandomizes the 16-bit source port.
    \item \textbf{Cache Poisoning:} With the source port identified, the remaining entropy is reduced to the 16-bit $TxID$,
    enabling the attacker to easily brute-force the remaining field and inject a definitive, malicious DNS response that
    is accepted by the resolver.
\end{enumerate}
To mitigate the SADDNS (Side-channel Attack on DNS) threat, security measures primarily focus on disrupting the ICMP-based side channel.
The core defense involves implementing randomized ICMP rate limiting within the network stack (notably in the Linux kernel),
which introduces statistical noise into the "Port Unreachable" response patterns. This process prevents attackers from reliably probing the state of UDP ports
on the resolver, preserving the effectiveness of Source Port Randomization (SPR) as a defense against DNS cache poisoning attacks.
Furthermore, the adoption of DNSSEC provides a structural defense by mandating cryptographic authentication of DNS records.
By doing so, any forged response is rejected for lack of a valid digital signature, regardless of whether the attacker successfully identifies the transaction parameters.

%% ==========================================
%% 3. TUDOOR : L'ATTAQUE LOGIQUE
%% ==========================================
\section{TuDoor Attack}\label{sec:contribution}

\subsection{TuDoor Attack Overview}

The \textbf{TuDoor attack}, presented at the 2024 IEEE Symposium on Security and Privacy, represents a significant advancement in the field of DNS security vulnerabilities.
Unlike previous attacks that primarily exploited protocol weaknesses or network side-channels (ICMP rate limiting), Tudoor targets \textbf{logic vulnerabilities} 
in the DNS response pre-processing mechanisms of recursive resolvers.

DNS response pre-processing is the phase where the resolver receives a packet, parses it, and validates it before exctracting the answer to cache or forward it.
Through the analyses of 28 different DNS software implementations, researchers discovered that while the fundamental rules of DNS are simple, 
the actual implementations are complex and often inconsistent in handling malformed packets.

The name "TuDoor" refers to a logic flaw that acts as a "hidden door" allowing attackers to inject malformed packets that should be rejected but are instead processed incorrectly,
triggering specific state transitions. These vulnerabilities enable three types of attacks: Denial of Service (DoS), Ressource Consumption, and the most critical one
that we will focus on in this paper, \textbf{Cache Poisoning}.

\subsection{Mecanism of the Vulnerability}

The core of the TuDoor cache poisoning attack relies on a specific logic vulnerability labeled $V_{CP}$. This vulnerability was prominently found in \textbf{Microsoft DNS},
a widely used entreprise resolver.

In a standard, secure implementation, when a recursive resolver opens a UDP socket to send a DNS query, it expects to receive a DNS \textit{response} packet in return.
A valid respond must have the \textbf{QR} (Query/Response) flag set to 1, indicating that it is indeed a response. If a packet arrives with the QR flag set to 0,
indicating it is not a response, the resolver should logically discard it.

However, the researchers found that Microsoft DNS fails to enforce this rule correctly during the pre-processing phase.
It allows a packet with the QR flag set to 0 to be received on the response socket, provided that the packet's "four-tuple" (source IP, source port, destination IP, destination port)
matches an active query.

Crucially, when Microsoft DNS receives this query packet on its response socket, it does not simply discard it. Instead, it treats it as a new incoming
question and initiates a recursive procedure to resolve the domain name contained within the malformed packet.
This behavior creates a deterministic \textbf{side-channel}: when an attacker sends a DNS query to a specific resolver port, the resolver initiates a new
resolution, thereby signaling to the attacker that the port is open and active.

\subsection{The DNS Cache Poisoning Attack Procedure}

Leveraging the $V_{CP}$ vulnerability, the TuDoor cache poisoning allows an off-path attacker to poison the DNS cache of a vulnerable resolver
without relying on global ICMP rate limits or other side-channels. The attack attack proceeds in the following steps:

\begin{figure}[!ht]
  \centering
  \includegraphics[width=\linewidth]{tudoor_poisoning.png}
  \caption{Attack steps of DNSPOISONING.}
  \label{fig:tudoor_poisoning}
\end{figure}

\begin{enumerate}
    \item \textbf{Setup and Trigger:} The attacker controls a nameserver ($NS_{Atk}$) for the domain $atkr.com$. The attacker
    initiates a DNS query to the target resolver for a victim domain (e.g., $vitm.com$).

    \item \textbf{Target State:} The target resolver sends a legitimate query to the authoritative nameserver for $vitm.com$.
    It opens a randomized source port ($Port_{RH}$) and waits for the response.

    \item \textbf{Probing Phase:} The attacker sends a series of spoofed DNS query packets to the target resolver,
    each with the QR flag set to 0. These packets are sent to different ports on the resolver, attempting to match the active source port ($Port_{RH}$).
    These queries ask the resolver to resolve a subdomain controlled by the attacker, encoding the guessed port number in the subdomain (e.g., $guess123.atkr.com$).

    \item \textbf{Hitting the Door:} When a spoofed packet hits the closed port, it is ignored. However, when the packet hits the correct active port ($Port_{RH}$),
    the logic vulnerability $V_{CP}$ is triggered. The resolver accepts the spoofed query packet.

    \item \textbf{The Oracle Signal:} Upon accepting the query $port12345.atkr.com$, the target resolver immediately sends a new DNS query
    to the attacker's nameserver ($NS_{Atk}$) to resolve it.

    \item \textbf{Port Identification:} The attacker observers the incoming query for $port12345.atkr.com$ on their own nameserver.
    This confirms that the open port on the resolver is indeed $12345$.

    \item \textbf{Cache Poisoning:} Now possessing the correct source port, the attacker only need to brute-force the 16-bit $TxID$.
    The attacker sends a flood of spoofed DNS response packets to the resolver, each with different $TxID$ values, 
    containing a malicious answer that redirects $vitm.com$ to a server under the attacker's control.

\end{enumerate}

\subsection{Comparative Analysis and Impact}

The TuDoor attack represents a significant evolution in DNS cache poisoning techniques compared to previous methods like Kaminsky and SADDNS.
The most notable feature of TuDoor is its speed. While SADDNS relies on ICMP rate limits that can take minutes to exploit, 
Tudoor is highly efficient.

\begin{itemize}
    \item \textbf{Deterministic vs Probabilistic:} SADDNS infers port based on the probability of ICMP packet supression.
    Tudoor receives a direct, deterministic signal from the resolver when the correct port is hit.
    \item \textbf{Execution Time:} Experimental results demonstrate that TuDoor can successfully identify the source port 
    and poison the cache in \textbf{less than 1 second} (average 425ms for Microsoft DNS), whereas SADDNS typically requires several minutes.
\end{itemize}

The implications of the TuDoor attack are profound. The researchers conducted internet-wide measurements and found that
the vulnerability is widespread. The cache poisoning vulnerability affects approximately 11.2\% of open DNS resolvers and expose them to instant cache poisoning attacks.
Overall, the different TuDoor attacks affects 23.1\% of the open DNS resolvers on the Internet, highlighting a significant security risk.
The primary defense involves deploying vendor-specific patches-addressing over 30 identified Common Vulnerabilities and Exposures that enforce strict validation of DNS headers
before any state is allocated for response processing. Furthermore, resolvers have been updated to implement stricter IPID randomization and more robust reassembly logic
to prevent off-path attackers from injecting malicious fragments. By disabling the processing of additional records in truncated responses and mandating a fallback to TCP
for large payloads, systems effectively close the "logic doors" exploited by the attack. Finally, as with previous cache poisoning threats,
the systematic implementation of DNSSEC provides a definitive layer of protection, as the cryptographic verification of records ensures 
that any successfully injected but forged data is discarded during the validation phase.

%% ==========================================
%% 4. DISCUSSION & CONCLUSION
%% ==========================================
\section{Discussion and Conclusion}\label{sec:discussion}

\subsection{Impact on DNS Security}

The TuDoor attack reveals a fundamental issue in DNS security: the gap between protocol specifications and their implementations.
In our opinion, this research is particularly valuable because it shifts the focus from protocol-level vulnerabilities to implementation-level logic flaws.
While previous attacks like Kaminsky and SADDNS exploited weaknesses in the DNS protocol itself or in operating system behaviors,
TuDoor demonstrates that even well-patched systems can harbor dangerous vulnerabilities in their parsing and pre-processing logic.

The widespread impact of TuDoor, affecting over 23\% of open DNS resolvers, underscores the urgent need for comprehensive security audits across all DNS implementations.
The fact that Microsoft DNS, a widely deployed enterprise solution, was vulnerable to such a fundamental logic flaw raises concerns about the security posture of other proprietary and open-source DNS software.
Moreover, the speed of the attack (less than 1 second) represents a critical escalation in the threat landscape, as it provides attackers with a near-instantaneous poisoning capability that can bypass many 
monitoring and detection systems designed to catch slower, more traditional attacks.

The TuDoor research also highlights the importance of defensive programming practices in critical infrastructure software.
The vulnerability stems from insufficient validation during the pre-processing phase, specifically the failure to properly enforce the QR flag check before allocating resources or triggering state transitions.
This suggests that many DNS implementations may benefit from adopting a more rigorous validation framework, where all packet fields are verified against RFC specifications before any processing occurs.
Additionally, the deployment of DNSSEC, while providing strong cryptographic guarantees, remains limited in practice, leaving many domains vulnerable to cache poisoning attacks regardless of the resolver's security posture.

\subsection{Conclusion}

The evolution of DNS security threats from Kaminsky through SADDNS to TuDoor illustrates a continuous arms race between attackers and defenders. 
While previous attacks targeted protocol weaknesses and system-level side channels, TuDoor's focus on implementation logic flaws opens a new chapter in DNS security research.

The key lesson from TuDoor is that secure protocol design alone is insufficient. Even with proper countermeasures like Source Port Randomization in place, 
inconsistent implementation of basic validation rules can undermine the entire security model. This emphasizes the need for thorough testing and verification of DNS software, particularly in how they handle edge cases and malformed packets.

Looking ahead, the DNS ecosystem must adopt a multi-layered defense strategy. Vendors should implement comprehensive validation frameworks during packet pre-processing, 
ensuring strict compliance with protocol specifications before any state allocation. The continued expansion of DNSSEC adoption remains essential, as cryptographic 
validation provides the only definitive protection against cache poisoning, regardless of the attack vector employed. The development of automated testing tools capable of detecting logic vulnerabilities across 
different DNS implementations would help identify these issues earlier in the development cycle.

\begin{thebibliography}{1}
\bibitem{tudoor}
X.~Li \emph{et al.}, ``TuDoor Attack: Systematically Exploring and Exploiting Logic Vulnerabilities in DNS Response Pre-processing with Malformed Packets,'' \emph{2024 IEEE Symposium on Security and Privacy (SP)}, pp. 4459-4477, 2024.

\bibitem{kaminsky}
D.~Kaminsky, ``DNS Vulnerability,'' Black Hat USA, 2008.

\bibitem{saddns}
K.~Qian \emph{et al.}, ``SADDNS: Exploiting Weakened Trust in DNS,'' \emph{ACM CCS}, 2020.

\bibitem{cloudflare}
M.~Vavruša and N.~Sullivan, "SAD DNS Explained," Cloudflare Blog, Nov. 13, 2020. Available: https://blog.cloudflare.com/sad-dns-explained/

\bibitem{IEEEhowto:kopka}
H.~Kopka and P.~W. Daly, \emph{A Guide to \LaTeX}, 3rd~ed.\hskip 1em plus
  0.5em minus 0.4em\relax Harlow, England: Addison-Wesley, 1999.
\end{thebibliography}

\end{document}